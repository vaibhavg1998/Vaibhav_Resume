\documentclass[11pt,a4paper,sans]{moderncv}

\usepackage{verbatim}

\moderncvstyle{classic}
\moderncvcolor{green}

\usepackage[utf8]{inputenc}

\usepackage[scale=0.8]{geometry}
\setlength{\hintscolumnwidth}{3cm}


% personal data
\firstname{Vaibhav}
\familyname{Gupta}
\address{Moh. Hatim Sarai, Near beej gaudawn,}{Railway Station Road, Sambhal,}{ Uttar Pradesh (244302)}
\email{vaibhavg1998@gmail.com}
\mobile{+91 8447359679}
\photo[64pt][0.4pt]{vaibhav_img}
\quote{"The power of the ideal is in the practical"   -Swami Vivekananda}

\begin{document}
\maketitle

\section{CAREER OBJECTIVE}
\cvitem{}{To further my studies to the highest level possible so
as to be able to orient, explore, calibrate, advance and
exploit my talents in order to work with the most
challenging and demanding organization with tougher
activities so as to boost, advance, improves and develops
my career in service to all people.}

\section{EDUCATION}

\begin{center}
	\begin{tabular}{ | m{3.2cm} | m{7.1cm}| m{2.2cm} | m{1.4cm} | m{1.8cm}| } 

	\hline
	Degree/Class & College/School & University/ Conducting Board & Passing Year & Pass Percentage/CGPA \\

	\hline
	B.Tech, Computer Science Engineering & Maharaja Agrasen Institute of Technology     , Rohini Sector-22, New Delhi (110086) & Indraprastha University & 2020 & 72\% \\

	\hline
	Senior Secondary XII (Science) & Sm Arya Public School, Punjabi Bagh, New Delhi (110026) & CBSE Board & 2016 & 90.40\% \\

	\hline
	Secondary X & St. Mary's Sec. School, Sambhal, Uttar Pradesh (244302)) & CBSE Board & 2014 & 9.00/10 \\

	\hline
	\end{tabular}
\end{center}

\section{PROJECTS}
\begin{enumerate}
	\item Image Classification using Deep Learning Model.
		\begin{itemize}
			\item Technologies used : Pytorch, Torchvision, Transfer learning, Google Colab, opencv, numpy, Convolutional Neural Networks (CNN).
			\item  Details: A image is given as an input to the pre- trained model (Resnet 152) and a vector of several classes was received as an output which contains the probability of the occurrence of the particular class. Analyzing this vector the name of the correct animal is predicted.
		\end{itemize}

	\item Programming an autonomous robot (FireBird 5).
		\begin{itemize}
			\item Technologies used : C Programming (Embedded C), Algorithms.
			\item Details : lists of animal location and the deposition locations (Habitat location) were communicated to the robot using Serial Communication. The task of the robot was to find the shortest path to traverse the graph (arena), pick the object and deposit the object to the destination.
		\end{itemize}
	\item Periodic Table using File Handling in C Programming Language.
		\begin{itemize}
			\item Technologies used : FILE Handling in C
Programming language.
			\item Details : In this project user can edit the data in the file and search the data using any of the property of the element (e.g. atomic number, atomic mass,symbol,name etc.).
		\end{itemize}
	\item Tic Tac Toe game development using JAVA Application Development.
		\begin{itemize}
			\item Technologies used : javax.swing package of Java, NetBeans IDE, Algorithms.
			\item Details : The important feature of the game is that even a single person can play the game with the computer.
Game is developed using algorithms in java. Accuracy of the computer to win is 92\%.
		\end{itemize}

\end{enumerate}

\section{TECHNICAL SKILLS}
\cvitem{}{
\begin{itemize}
	\item C++ Programming Language (Advanced)
	\item C Programming            (Advanced)
	\item Data Structures and Algorithms (Advanced)
	\item Deep Learning            (Intermediate)
	\item Java					   (Intermediate)
	\item Open CV                  (Beginner)
	\item Python                   (Beginner)
\end{itemize}
}

\section{TRAINING AND INTERNSHIPS}

\section{RESEARCH PUBLICATIONS}

\end{document}